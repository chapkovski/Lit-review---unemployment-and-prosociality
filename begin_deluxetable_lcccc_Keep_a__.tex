\begin{deluxetable}{lcccc}

%% Keep a portrait orientation

%% Over-ride the default font size
%% Use Default (12pt)

%% Use \tablewidth{?pt} to over-ride the default table width.
%% If you are unhappy with the default look at the end of the
%% *.log file to see what the default was set at before adjusting
%% this value.

%% This is the title of the table.
\tablecaption{OLS Results}

%% This command over-rides LaTeX's natural table count
%% and replaces it with this number.  LaTeX will increment 
%% all other tables after this table based on this number
\tablenum{1}

%% The \tablehead gives provides the column headers.  It
%% is currently set up so that the column labels are on the
%% top line and the units surrounded by ()s are in the 
%% bottom line.  You may add more header information by writing
%% another line between these lines. For each column that requries
%% extra information be sure to include a \colhead{text} command
%% and remember to end any extra lines with \\ and include the 
%% correct number of &s.
\tablehead{\colhead{} & \colhead{Employed} & \colhead{Unemployed} & \colhead{} & \colhead{} \\ 
\colhead{} & \colhead{(coef.)} & \colhead{(p)} & \colhead{coef.} & \colhead{p} } 

%% All data must appear between the \startdata and \enddata commands
\startdata
job for everyone who wants one  & -0.115 & 0.003 & 0.186 & 0.029 \\
adequate healthcare for the sick  & -0.019 & 0.452 & 0.114 & 0.035 \\
reasonable standard of living for the old & 0.031 & 0.235 & 0.131 & 0.019 \\
reasonable standard of living for the unemployed & -0.193 & 0 & 0.388 & 0 \\
sufficient childcare services for working parents & 0.025 & 0.427 & 0.155 & 0.024 \\
paid leave from those who care for sick family members & 0.006 & 0.864 & 0.186 & 0.008 \\
\enddata

%% Include any \tablenotetext{key}{text}, \tablerefs{ref list},
%% or \tablecomments{text} between the \enddata and 
%% \end{deluxetable} commands

%% General table comment marker
\tablecomments{adsf}

%% General table references marker
\tablerefs{asdf}

\end{deluxetable}